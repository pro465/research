\documentclass{article}
\usepackage[utf8]{inputenc}
\usepackage[english]{babel}
\usepackage[]{amsmath}
\usepackage[]{amsthm} %lets us use \begin{proof}
\usepackage[]{amssymb} %gives us the character \varnothing

  \newtheorem{definition}{Definition}[section]
\newtheorem{theorem}{Theorem}[section]
\newtheorem{corollary}{Corollary}[theorem]
\newtheorem{lemma}[theorem]{Lemma}

\title{Project Euler - Problem 103}
\author{Proloy Mishra}
\date{June 5, 2024}

\begin{document}
   \maketitle
   \section{important}
   \begin{theorem}[]
    there are no non-trivial solutions to the equation $ S * C = \mathbf{0}, $ where
     $$ C =
     \begin{bmatrix}
        0 & 1 & 0 \\
        1 & -4 & 1 \\
        0 & 1 & 0
     \end{bmatrix}
        , $$ and $ * $ denotes the convolution operator with wrapping around the edges.
    \end{theorem}
    \begin{proof}
      To start, we note 2 properties of $ * $: \\
         $$ A * C + B * C = (A + B) * C, $$
         $$  \left[
\begin{array}{ccc}
   n & \cdots & n \\
   \vdots & \ddots & \vdots \\
   n & \cdots & n
\end{array}
\right] * C = \mathbf{0} $$
      these combined, imply that we can "reduce" a move such that there exists at least one 0 entry, and all entries are nonegative.
      now, let's say that there exists a non-trivial (not every entry being equal) solution. Then we can reduce it. now we get a move where:\\
       \indent 1. all of its entries are nonnegative and there exist at least one 0 entry. (since we reduced it)\\
       \indent 2. at least one of its entry is positive. (otherwise it's trivial) \\
     these 2 imply there exists a 0 entry such that at least one of its non-diagonal neighbours is positive.
     now, we can see that to be a solution of the equation, it must be that the convolution of the 0 entry along with its non-diagonal neighbours must be 0,
      which basically means
       $$ a + b + c + d = 0, $$
      where $a, b, c, $ and  $d$ are the neighbouring entries of the 0 entry
      but this is impossible, as at least one of them is positive, and the rest is non-negative, so the sum is always positive.
    \end{proof}
    \begin{proof}
        Let there be a toppling sequence $ p_1, \cdots, p_n $ of cells and a specific cell $ p_i $. Let's say it has $ k $ of its $ 4 $ orthogonal neigbours already toppled, giving it an advantage of $ k $, making the minimum required number of stones in it before the cycle be $ 4 - k. $ If we move it to the front of the sequence so that it's the first cell to be toppled, its minimum requirement now becomes $ 4, $ increasing the total requirement by $ k. $\\ 
       However, each of its $ k $ neighbours now get an additional advantage of $ 1 $ from another one of their neighbours (itself, more specifically $ p_i $) getting toppled, hence decreasing the total minimum requirement by $ 1 \cdot k = k $, thus making the net change from the ordering change $ 0 $
    \end{proof}
\end{document}
