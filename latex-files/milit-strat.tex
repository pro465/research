\documentclass{article}
\usepackage[utf8]{inputenc}
\usepackage[english]{babel}
\usepackage[]{amsmath}
\usepackage[]{amsthm} %lets us use \begin{proof}
\usepackage[]{amssymb} %gives us the character \varnothing

\newtheorem{definition}{Definition}[section]
\newtheorem{theorem}{Theorem}[section]
\newtheorem{corollary}{Corollary}[theorem]
\newtheorem{lemma}[theorem]{Lemma}

\title{The Torpedo Problem}
\author{Moja}
\date{12 February, 2025}

\begin{document}
    \maketitle
    \section{Definition}
    \begin{definition}[The Problem]
        Let there be a submarine with initial position $P_0 \in \mathbb{Z}$ and speed $v \in \mathbb{Z}$, both of which are unknown to us. 
        Find a strategy to drop torpedoes $f: \mathbb{N} \to \mathbb{Z}$ s.t. no matter what $P_0$ and $v$ are, the submarine gets destroyed in finite time. \\
        That is, for every choice of $P_0$ and $v$, there exists a time $t \in \mathbb{N}$ s.t. 
            $$ f(t) = P_0 + vt. $$
    \end{definition}

    \section{Solution}
    \begin{theorem}
	Let $g: \mathbb{N} \to \mathbb{N}$ be defined by: 
             $$ g(0) = 0; $$
	    $$ g(2x+y) = 2g(\lfloor \frac{x}{2} \rfloor) + y, y \in \{ 0, 1 \} $$ 
	and $f: \mathbb{N} \to \mathbb{Z}$ by:
	    $$ f(4x+2y+z) = (-1)^yg(\lfloor \frac{x}{2} \rfloor)+(-1)^z(4x+2y+z)g(x), y,z \in \{ 0, 1 \} $$
	Then, the strategy given by the function $f$ is a solution to the aforementioned problem.
    \end{theorem}
    \begin{proof}
	    To prove this, note that the function $h: \mathbb{N} \to \mathbb{Z}^2$ defined by 
	    $$ h(4x+2y+z) = ((-1)^yg(\lfloor \frac{x}{2} \rfloor), (-1)^zg(x)), y,z \in \{ 0, 1 \} $$
	    is bijective. Indeed, $g(x)$ just takes every odd positioned bit of $x$ to create its result, so the whole thing is just separating the odd- 
	    and even-positioned bits to create two signed numbers, with their first bit deciding their signs. Thus, to invert the operation, 
	    just interleave the bits and signs of the numbers.\\

	    Also, note that $$f(t) = \text{fst}(h(t)) + \text{snd}(h(t))t$$ where $\text{fst, snd}: \mathbb{Z}^2 \to \mathbb{Z}$ are defined by 
	    $$ \text{fst}((x, y)) = x, $$ and $$ \text{snd}((x, y)) = y. $$ \\
	    Now, say the submarine has initial position $P_0$ and velocity $v.$ Since we just proved $h$ is bijective we know there is a value $t$ for 
	    which $$h(t) = (P_0, v). $$ But that means 
\begin{equation}
\begin{aligned}
f(t) &= \text{fst}(h(t))+\text{snd}(h(t))t \\
&= P_0 + vt,
\end{aligned}
\end{equation}
 which is what we wanted to prove.
    \end{proof}
\end{document}
