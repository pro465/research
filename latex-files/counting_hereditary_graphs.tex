\documentclass{article}

\usepackage{amsmath}
\usepackage{amsthm}
\usepackage{geometry}
\usepackage{hyperref}
\geometry{
  paper=a4paper,
  margin=70pt,
}

\title{Counting Hereditary Graphs}
\date{2024-05-22}
\author{Proloy}

\begin{document}
  \maketitle

  A few days/weeks ago (the author doesn't remember the exact time), TARDIInsanity posted in the Esolangs Discord channel a new maths problem. 
  It was actually designed to get insights into a more complicated problem, but ended up being interesting in its own right.

  \section{The problem}
  To get to the problem, we need to learn what DWD graphs are first. They stand for ``Distance Well-Defined graphs", 
  although the author would want to call them DDWD for Directed DWDs.
  They are directed graphs where the distance between any two specific nodes is the same regardless of the path taken to go from one to another.
  the distance of any path is defined to be the sum of the distances of the edges travelled in the path, where
  the distance along a directed edge is:\\
  \indent 1 if it's directed along the direction of the edge.\\
  \indent -1 if it's directed opposite the direction of the edge.\\

  Also, the graph must be connected.

  The original problem was to count the number of (D)DWDs given $n$ nodes. But that proved too diffcult for us, 
  so TARDIInsanity came up with an easier problem.
  
  \noindent The new problem (and the problem we will solve in this article) is, to quote them: \\
  \emph{
  \indent
     How many ``hereditary" DWD graphs exist, as a function of the depth (the maximum distance)?\\
  \indent ``hereditary" means: \\
  \indent \indent  a. there is exactly one leaf node (one node with 0 outgoing edges)\\
  \indent \indent  b. all nodes have exactly 0 or 2 incoming edges\\
  \indent \indent  c. all nodes with 0 incoming edges are equidistant from the (singular) leaf.
  }

  \section{Layers}
    A (D)DWD graph can be easily arranged to form layers with 0 distance between the nodes in any particular layer. The reader is requested to imagine some examples
    as the author is too lazy (and perhaps embarrassed) to create terrible drawings to communiciate his point.

    Note that all the nodes in a layer can only have outgoing (or incoming) edges to at most one layer. 
    This is because if the nodes have outgoing (or incoming) edges to more than two layer, the distance is no longer well defined.

    Together, these give the reason for the name choice of ``hereditary (D)DWD graphs". Each layer could be considered a generation, and 
    the two nodes to which the incoming edges of each node are coming from could be considered their parents. Hence the graphs could be considered family "trees" 
    (actually graphs since\emph{... ahem...} genetically pure children are allowed.)

    \subsection{A related but simpler (sub-)problem}
      Let's focus only on the connection between two layers. The question we ask is: 
      \emph{\paragraph{}
      How many ways are there for $n$ parents to give birth to $k$ children, 
      with no parent being childless and no two children being siblings?}

      (Note that distinct pairs of parents give birth to distinct children.)

      Let $c(n, k)$ denote the answer to the above question. Then:
      
        $$c(n, k) = \sum_{i=0}^{n} (-1)^{n-i} {n \choose i} {{i \choose 2} \choose k}. $$
        
      \begin{proof}
        We proceed using induction.
        \subparagraph{Base case} The amount of undercounting for $n-2$ non-isolated nodes in $$ {{n \choose 2} \choose k} - n {{n \choose 2} \choose k} $$ is $ {c(n-2, k)}, $ which is counted exactly once in $$ (-1)^{n-(n-2)} {n \choose n-2} {{n-2 \choose 2} \choose k} . $$
        \subparagraph{Inductive case} Let $a(n, k, j)$ be the amount of overcounting or undercounting with the partial sum involving only terms with $i > j$. Then,
        \begin{align*}
          a(n, k, j) &= c(j, k) \sum_{i=j+1}^{n} (-1)^{n-i} {n-j \choose n-i} \\
                     &=c(j, k) \sum_{i=0}^{n-j-1} (-1)^{i} {n-j \choose i} \\
                     &= c(j, k) (-1)^{n-j-1} {n-j-1 \choose n-j-1} \\
                     &= c(j, k) (-1)^{n-j-1}
        \end{align*}
        (For the third step, see \ref{f1}.)
        Thus, the compensation is $$ -a(n, k, j) = c(j, k) (-1)^{n-j}, $$ which is counted exactly once in $$(-1)^{n-j} {n \choose j} {{j \choose 2} \choose k}. $$
        \end{proof}

    \subsection{Moving up and close}
    Let's make it harder and closer to the problem we are trying to solve. We ask ourselves:
    \emph{
      \paragraph{}
      Given 3 multisets $A$, $B$, and $C$ of nodes,
      how many ways are there for the nodes in $A$ and $B$ to give birth to $C$, with no childless node in B?\\
    }
    
    \noindent A multiplicity of a node in a multiset represents the total number of its siblings plus itself. The author hopes the reader can figure out how this is 
    helpful in the quest to solve the final problem.

    A multiset of nodes can be represented by a multiset of multiplicities, where each element represents a single group of siblings.
    
    Let $f(A, B, C)$ denote the answer to the above question. Then,
    \begin{align*}
      f(A, [b]t :: B, C) &= 
    \end{align*}

\section{References}
\paragraph{}
\label{f1} \url{https://proofwiki.org/wiki/Alternating_Sum_and_Difference_of_r_Choose_k_up_to_n}

\end{document}
